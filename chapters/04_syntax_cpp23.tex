\chapter{The Syntax in C++23}

\section{Threads, Tasks, Coroutines}

\subsection{C++11 Threads}

\raggedbottom
\begin{codebox}[]{\href{https://godbolt.org/z/}{\ExternalLink}}
\footnotesize An example of the use of a C++11 thread.
\tcblower
\cppfile{code_examples/cpp23/thread1.cpp}
\end{codebox}

\raggedbottom
\begin{codebox}[]{\href{https://godbolt.org/z/}{\ExternalLink}}
\footnotesize An example of the use of a C++11 thread.
\tcblower
\cppfile{code_examples/cpp23/thread2.cpp}
\end{codebox}

\subsection{C++20 Threads}

\raggedbottom
\begin{codebox}[]{\href{https://godbolt.org/z/}{\ExternalLink}}
\footnotesize An example of the use of jthread.
\tcblower
\cppfile{code_examples/cpp23/jthread1.cpp}
\end{codebox}

\raggedbottom
\begin{codebox}[]{\href{https://godbolt.org/z/}{\ExternalLink}}
\footnotesize An example of the use of jthread.
\tcblower
\cppfile{code_examples/cpp23/jthread2.cpp}
\end{codebox}

\raggedbottom
\begin{codebox}[]{\href{https://godbolt.org/z/}{\ExternalLink}}
\footnotesize An example of the use of jthread.
\tcblower
\cppfile{code_examples/cpp23/jthread3.cpp}
\end{codebox}

\raggedbottom
\begin{codebox}[]{\href{https://godbolt.org/z/}{\ExternalLink}}
\footnotesize An example of the use of jthread.
\tcblower
\cppfile{code_examples/cpp23/jthread4.cpp}
\end{codebox}

\raggedbottom
\begin{codebox}[]{\href{https://godbolt.org/z/}{\ExternalLink}}
\footnotesize An example of the use of jthread.
\tcblower
\cppfile{code_examples/cpp23/jthread5.cpp}
\end{codebox}

\raggedbottom
\begin{codebox}[]{\href{https://godbolt.org/z/}{\ExternalLink}}
\footnotesize An example of the use of jthread.
\tcblower
\cppfile{code_examples/cpp23/jthread6.cpp}
\end{codebox}

\subsection{Tasks}

\subsection{Coroutines}

In C++23 and subsequent versions, we hope to see some standard syntax for defining coroutines (cf. P2502). 

\raggedbottom
\begin{codebox}[]{}
\footnotesize An example of the use of coroutines, which currently does not compile in GCC 12.2.
\tcblower
\cppfile{code_examples/cpp23/coroutines1.cpp}
\end{codebox}

Unfortunately, defining the coroutine in C++20 take some more effort. In particular, it requires:
\begin{itemize}
\item defining the behaviour of the coroutine, which is known as a \cpp{promise} (different from \cpp{std::promise}), and requires one returns the type used to access the state of the coroutine on the heap, which is known as the handle,  
\item defining how to store the state of the coroutine on the heap, using template class \cpp{std::coroutine_handle} parametrized by the promise.
\end{itemize}
Clearly, one needs to declare one, define the other, and then return to declare the first one. We will see how to do this later.

In terms of using the coroutine, there are three new keywords:
\begin{itemize}
\item \cpp{co_await awaiter} suspends computation and block the co-routine until the computation is resumed by another co-routine calling ``resume'' method of the present coroutine. In the process, it tests whether it is possible to suspend the computation using an awaiter such as \cpp{std::suspend_always{};} (or an awaitable object, more generally, as discussed below) and, if so, saves all local variables to a heap-allocated handle.
\item \cpp{co_yield} yields a value and suspends computation as above, and  
\item \cpp{co_return} returns a value. (There is no notion of an optional return type in-built.)
\begin{end}

A difficulty in using coroutines is the fact that the coroutine may live longer than the scope it has been called from. It is hence \emph{not} advisable to pass by reference, except perhaps \cpp{std::ref} or \cpp{std::cref}. One can either pass by value or pass, e.g., \cpp{std::unique_ptr}:

\raggedbottom
\begin{codebox}[]{}{}
\footnotesize An example of the use of coroutines, which currently does not compile in GCC 12.2.
\tcblower
\cppfile{code_examples/cpp23/coroutines2.cpp}
\end{codebox}

Below, we will see two examples of the use of coroutines that compile with GCC 11 and 12, when you enable -std=c++2b -fcoroutines, but illustrate that the current use of the template class \cpp{std::coroutine_handle<promise>} is a bit unwieldy. 
First, we need to be able to define a promise class, which defines the behaviour of the coroutine by implementing methods:
\begin{itemize}
\item \cpp{coroutine get_return_object()} is called to inialize the coroutine and create the coroutine handle, which can be the rather formulaic \cpp{coroutine_handle::from_promise(*this)};} 
\item \cpp{std::suspend_always initial_suspend()}, suggests whether the coroutine starts right after initialization (\cpp{std::suspend_never()} ) or upon resumption (\cpp{std::suspend_always()}). (Both awaiterers are described below.)
\item \cpp{std::suspend_always final_suspend() noexcept}, which can be rather formulaic \cpp{std::suspend_always()}
\item \cpp{void return_void()} or \cpp{void return_value(const auto& value)}, which is called upon reaching the end of the coroutine and upon reaching \cpp{co_return}. The latter (\cpp{return_value}) often just stores the result locally. 
\item \cpp{void unhandled_exception()}, which can be rather formulaic \cpp{std::terminate()}, or can save the exception via \cpp{std::current_exception()}.
\end{itemize}

The promise class is instantiated for each instance of the coroutine, and its methods are called as follows:

\raggedbottom
\begin{codebox}[]{}
\footnotesize An schema of the coroutine, in terms of its calls of the methods of the promise class.
\tcblower
\cppfile{code_examples/cpp23/coroutines_schema.h}
\end{codebox}

Once we have a promise class, we can specialize template class \cpp{std::coroutine_handle}, which can be seen as the equivalent of a pointer and its method ``destroy'' as equivalent to a ``free'', and use the handle specialized to our own promise class to define a promise class:

\raggedbottom
\begin{codebox}[]{\href{https://godbolt.org/z/1nYdMPh3z}{\ExternalLink}}
\footnotesize An example of the use of coroutines.
\tcblower
\cppfile{code_examples/cpp23/coroutines3.cpp}
\end{codebox}

Sometimes, we also store the promise type in a \cpp{promise_type} type member, and disable (\cpp{= delete}) copy and move constructors. 

\subsubsection{Awaiters}

Finally, let us consider awaiters, which can be called when a coroutine is suspended or resumed. 
Key methods of an awaiter include:
\begin{itemize}
\item \cpp{await_ready()} is called immediately before suspension of a coroutine. If it returns \cpp{true}, the coroutine will not be suspended. 
\item \cpp{await_suspend(handler)} is called immediately after the suspension of the coroutine. The \cpp{handler} of type \cpp{std::coroutine_handle} can be used to pass the state of the coroutine (e.g., to another thread). 
\item \cpp{await_resume()} is called when the coroutine is resumed after a successful suspension. If it returns a value, this will be returned by the \cpp{co_await} routine. 
\end{itemize}
The awaiters we have seen so far (\cpp{std::suspend_never()} and \cpp{std::suspend_always()}) returned boolean constants in \cpp{await_ready()}:

\raggedbottom
\begin{codebox}[]{\href{https://godbolt.org/z/}{\ExternalLink}}
\footnotesize An example of two standard awaiters.
\tcblower
\cppfile{code_examples/cpp23/awaiters1.cpp}
\end{codebox}

By defining \cpp{await_transform()} in the promise type, the compiler will use \cpp{co_await promise.await_transform(<expr>)} instead of any call of \cpp{co_await <expr>} in the coroutine. 

\raggedbottom
\begin{codebox}[]{\href{https://godbolt.org/z/}{\ExternalLink}}
\footnotesize An example of the use of coroutines.
\tcblower
\cppfile{code_examples/cpp23/coroutines4.cpp}
\end{codebox}

Eventually, in C++23 and C++23, the support for message-passing architectures based on \url{https://github.com/lewissbaker/cppcoro} should be available. For a small sample, see a custom implementation here:

\raggedbottom
\begin{codebox}[]{\href{https://godbolt.org/z/}{\ExternalLink}}
\footnotesize An example of the use of coroutines.
\tcblower
\cppfile{code_examples/cpp23/coroutines5.cpp}
\end{codebox}

For more nice examples, see Boost Asio, e.g.,
\url{https://www.boost.org/doc/libs/1_78_0/doc/html/boost_asio/example/cpp20/channels/throttling_proxy.cpp}, 
as discussed, e.g., at
\url{https://www.youtube.com/watch?app=desktop&v=ZNttI_WswMU&ab_channel=ACCUConference}.
For more details, see Simon Toth's Complete guide at \url{https://itnext.io/c-20-coroutines-complete-guide-7c3fc08db89d}.
For the technical specification, see \url{https://github.com/GorNishanov/coroutines-ts}.

\section{Synchronisation Primitives}

\subsection{Barrier}

\raggedbottom
\begin{codebox}[]{\href{https://godbolt.org/z/}{\ExternalLink}}
\footnotesize An example of the use of a barrier.
\tcblower
\cppfile{code_examples/cpp23/barrier1.cpp}
\end{codebox}

\raggedbottom
\begin{codebox}[]{\href{https://godbolt.org/z/}{\ExternalLink}}
\footnotesize An example of the use of a barrier.
\tcblower
\cppfile{code_examples/cpp23/barrier2.cpp}
\end{codebox}

\subsection{Atomic Variables}

\raggedbottom
\begin{codebox}[]{\href{https://godbolt.org/z/}{\ExternalLink}}
\footnotesize An example of the use of atomic variables.
\tcblower
\cppfile{code_examples/cpp23/atomic1.cpp}
\end{codebox}

\raggedbottom
\begin{codebox}[]{\href{https://godbolt.org/z/}{\ExternalLink}}
\footnotesize An example of the use of atomic variables.
\tcblower
\cppfile{code_examples/cpp23/atomic2.cpp}
\end{codebox}

\raggedbottom
\begin{codebox}[]{\href{https://godbolt.org/z/}{\ExternalLink}}
\footnotesize An elaborate example of the use of atomic variables.
\tcblower
\cppfile{code_examples/cpp23/atomic3.h}
\end{codebox}

\subsection{Mutexes}

\raggedbottom
\begin{codebox}[]{\href{https://godbolt.org/z/}{\ExternalLink}}
\footnotesize An example of the use of a barrier.
\tcblower
\cppfile{code_examples/cpp23/mutex1.cpp}
\end{codebox}

\raggedbottom
\begin{codebox}[]{\href{https://godbolt.org/z/}{\ExternalLink}}
\footnotesize An example of the use of a barrier.
\tcblower
\cppfile{code_examples/cpp23/mutex2.cpp}
\end{codebox}

\section{Algorithms in the Standard Template Library}

\subsection{For Each}

\raggedbottom
\begin{codebox}[]{\href{https://godbolt.org/z/}{\ExternalLink}}
\footnotesize An example of the use of for each.
\tcblower
\cppfile{code_examples/cpp23/for_each_code_cpp20.h}
\end{codebox}

\raggedbottom
\begin{codebox}[]{\href{https://godbolt.org/z/}{\ExternalLink}}
\footnotesize An example of the use of for each.
\tcblower
\cppfile{code_examples/cpp23/for_each_code_parallel.h}
\end{codebox}

\subsection{Reduce}

\raggedbottom
\begin{codebox}[]{\href{https://godbolt.org/z/}{\ExternalLink}}
\footnotesize An example of the use of reduce.
\tcblower
\cppfile{code_examples/cpp23/reduce_code.h}
\end{codebox}

\subsection{Merge}

\raggedbottom
\begin{codebox}[]{\href{https://godbolt.org/z/}{\ExternalLink}}
\footnotesize An example of the use of a merge.
\tcblower
\cppfile{code_examples/cpp23/merge_par_code.h}
\end{codebox}
