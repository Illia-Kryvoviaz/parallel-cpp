\chapter{The Syntax in OpenMP}

OpenMP is a specification for parallel programming of shared-memory systems in Fortran, C, and C++. 
The current version of the specification can be downloaded from \url{https://www.openmp.org/wp-content/uploads/OpenMP-API-Specification-5-2.pdf}, with many examples at \url{https://github.com/OpenMP/Examples}.

The specification is built on top of the fork-join model of parallel execution (sériově-paralelní model uspořádání vláken),
but generally does not provide any guarantees as to how a particular directive or function is implemented. 
This also means that running on a different hardware may result in different order of execution of floating-point operations and vastly different performance.  
Prime implementations include 
\begin{itemize}
\item libgomp (GOMP) for GCC, 
\item libomp for clang, and 
\item liomp5 (IOMP) for ICC/Clang.
\end{itemize} 

OpenMP specifies preprocessor directives (pragmas) and a library of functions exported via \cpp{omp.h}.
Ideally, one and the same program written with OpenMP should be possible to run as serial code, or with any number of threads. 
Many OpenMP programs use only the preprocessor directives, which makes it possible to compile them as serial code even without OpenMP-aware compiler. 

There is a long history of the evolution of OpenMP since 1997, from a 50-page long specification of OpenMP 1.0, through 150-page version 3.0 (introducing ``tasks'') and 250-page version 4.0 (adding GPGPU support), to much longer versions recently. 
Traditional implementations of OpenMP have been rather closely built on top of Pthreads, 
which results in the lack of fine-grained scheduling, 
memory management, network management, signaling, etc.
The lack of fine-grained scheduling notably meant the lack of user-level threads (co-routines)
and the lack of queries as to the number of hardware threads utilized by other processes, 
which often results in high overhead when the number of threads (across all processes!) increases above the number of hardware threads supported.
Since OpenMP 5.0, the distinction between threads and tasks has been erased and thread teams are also cast into tasks. There are now also OpenMP implementations over lightweight threads, notably (BOLT is OpenMP over Lightweight Threads, https://www.bolt-omp.org/). 

\section{Threads, Tasks, Coroutines}  

\subsection{OpenMP Task Region}

Initially, OpenMP application starts with a single thread (initial/master thread). This can spawn parallel regions, typically with multiple threads in a thread pool (``thread team'') in the fork-join manner. 
This means that in any thread, one can either wait for all the ``sibling'' threads to finish (``join them'') or spawn a further, nested parallel region.

A key construct in OpenMP is thus \cpp{#pragma omp parallel}, which delineates a parallel region and opens a ``team of OpenMP threads'', which could be seen as a thread pool of threads or user-level threads. 
By default, the number of threads is set based on the available hardware threads, but this can be affected by the environment variables (\cpp{OMP_NUM_THREADS}) and modifiers of the pragma (\cpp{num_threads(2)}) and function calls (\cpp{omp_set_num_threads()}). 

\raggedbottom
\begin{codebox}[]{\href{https://godbolt.org/z/bdP9o8rov}{\ExternalLink}}
\footnotesize An example of the use of OpenMP parallel regions.
\tcblower
\cppfile{code_examples/openmp/parallel1.cpp}
\end{codebox}

The construct can take a number of modifiers, including:
\begin{itemize}
\item \cpp{num_threads}: number of threads to use in the team
\item \cpp{private}(list of variables): those variables will be private to each thread 
\item \cpp{firstprivate}(list of variables): those variables will be private to each thread, but initially their value will be copied from the master thread using the default copy constructor. 
item \cpp{lastprivate}(list of variables): those variables will be private to each thread, but at the end, their value will be copied to the master thread using the default copy constructor. 
\item \cpp{shared}(list of variables): these variables will be shared between the master thread and all threads in the new team. It is the programmers responsibility to keep the variables constructed as long as the parallel region is running
\item \cpp{default}: values of \cpp{private}, \cpp{firstprivate}, \cpp{shared}, \cpp{none} suggest what should be the default behaviour for variables not listed above. The default is \cpp{shared}, which is suboptimal from both performance and thread-safety perspective. It is wise to issue \cpp{default(none)}. 
\item \cpp{reduction}(reduction-identifier : list) suggests that variables in list should be treated as shared, when they are used by the function \cpp{reduction-identifier}, which could also take the special values of $+, -, *, \&, |, ^, ||, \max, \min$. The list can include array elements and, when \cpp{reduction-identifier} is a static function of a class, accessible data objects of the object. \label{sec:reduction1}
\item \cpp{proc_bind}: values of \cpp{master} and \cpp{close} and \cpp{spread} suggest how far from the master thread should be executed the new threads (same core, close in non-uniform architectures, as far as possible in non-uniform architectures).
%\item allocate([allocator :] list)
\end{itemize}

Whether the nested parallel regions create their own thread teams or use the existing thread teams depends on   
settings that we can affect through \cpp{omp_set_nested}, or environment variables \cpp{OMP_NESTED} (which can be true or false) and \cpp{OMP_MAX_ACTIVE_LEVELS}, which controls the maximum number of nested active parallel regions.
See, for example:

\raggedbottom
\begin{codebox}[]{\href{https://godbolt.org/z/Ejh5EfPqj}{\ExternalLink}}
\footnotesize An example of the use of OpenMP parallel regions.
\tcblower
\cppfile{code_examples/openmp/parallel2.cpp}
\end{codebox}

Since version 5.0, one can also utilize the \cpp{teams} construct on its own, optionally with a target (such a GPGPU). 

\subsection{Threads and their Sizing}

As has been mentioned above, ideally one and the same program written with OpenMP should be possible to run as serial code, or with any number of threads. 
This requires sizing the work in each thread depending on the number of threads, such as in:

\raggedbottom
\begin{codebox}[]{\href{https://godbolt.org/z/vTx9Mxzno}{\ExternalLink}}
\footnotesize An example of the use of OpenMP parallel regions.
\tcblower
\cppfile{code_examples/openmp/parallel3.cpp}
\end{codebox}

\subsection{Sections}

An alternative, non-iterative structuring of the code is possible with sections.
Each section is a block of code executed by one thread of the current thread team (corresponding to the innermost enclosing parallel region). 
One can use \cpp{private}, \cpp{firstprivate}, \cpp{lastprivate}, and \cpp{reduction} modifiers, similar to the parallel construct. 
There is an implied barrier of the sections region, unless eliminated by a \cpp{nowait} clause.
See, for example:

\raggedbottom
\begin{codebox}[]{\href{https://godbolt.org/z/PrEPzhsGn}{\ExternalLink}}
\footnotesize An example of the use of OpenMP sections.
\tcblower
\cppfile{code_examples/openmp/section1.cpp}
\end{codebox}

\raggedbottom
\begin{codebox}[]{\href{https://godbolt.org/z/fTsv1qz4K}{\ExternalLink}}
\footnotesize An example of the use of OpenMP sections.
\tcblower
\cppfile{code_examples/openmp/section2.cpp}
\end{codebox}

\subsection{Tasks}

The closest to a coroutine in OpenMP is the concept of a task. While it does not come with a promise of an implementation with a user-level thread library (cf. Argobots, Converse threads,	Qthreads,	MassiveThreads, Nanos++, Maestro, GnuPth, StackThreads/MP, Protothreads, Capriccio, StateThreads, TiNy-threads, etc), it often is implemented thus.  

\raggedbottom
\begin{codebox}[]{\href{https://godbolt.org/z/TM5WjPvT9}{\ExternalLink}}
\footnotesize An example of the use of OpenMP tasks.
\tcblower
\cppfile{code_examples/openmp/task1.cpp}
\end{codebox}

\subsection{Kernels}

There is a nascent support for the use of GPGPUs via \cpp{target} construct.
While the block following the target construct can be arbitrary, in principle, most GPUs are not able to support arbitrary code. Specifically, there are usually no synchronization primitives available and no coherence among L1 caches. Often, block requiring anything beyond the basic arithmetics will execute only on one core (``stream multiprocessor'') of the GPGPU. Ideally, one should combine the offloading of the code to the GPGU (via the target), across multiple cores (via teams construct), and then across the code. This can get quite non-trivial:

\raggedbottom
\begin{codebox}[]{\href{https://godbolt.org/z/TM5WjPvT9}{\ExternalLink}}
\footnotesize An example of the use of OpenMP targets and teams.
\tcblower
\cppfile{code_examples/openmp/target1.h}
\end{codebox}

\raggedbottom
\begin{codebox}[]{}{}
\footnotesize Another example of the use of OpenMP targets and teams.
\tcblower
\cppfile{code_examples/openmp/target2.h}
\end{codebox}

We could also use the nowait clause to make the target constructs asynchronous, resembling other OpenMP tasks,
but that does not match the ``lockstep'' execution on the GPGPUs. 

Compiling OpenMP with offloading is non-trivial. As has been mentioned in the introduction to this chapter, there are multiple implementations of OpenMP. Most likely, you have used \verb|gcc -fopenmp|, so far. Depending on what GPGPUs you wish to target, you may need to switch:

%Intel compilers are among those supporting SPIR-V, which is a modern binary low-level intermediate language used across OpenCL, OpenGL, and Vulkan.

% g++-mp-12

\begin{itemize}
\item NVIDIA CUDA Compiler Driver NVCC may be the easiest to use with NVIDIA hardware. For A100 of the RCI cluster, use cc80: \verb|nvcc –mp=gpu -gpu=cc80|
\item clang/LLVM compilers uses \verb|clang++ -fopenmp -fopenmp-targets=<target triple>| where the triple is documented at \url{https://llvm.org/doxygen/Triple_8h_source.html} such as \verb|clang++ -fopenmp -fopenmp-targets=nvptx64 -O3|
\item AMD ROC is built on top of Clang, so starting with \verb|clang -fopenmp -offload-arch=gfx908| is a good idea. In more detail,  
\verb|-target x86_64-pc-linux-gnu -fopenmp -fopenmp-targets=amdgcn-amd-amdhsa -Xopenmp-target=amdgcn-amd-amdhsa -march=<gpu-arch>|
\item Intel Compiler Collection is, since 2021, also built on top of Clang, except still uses the Intel OpenMP library, so you start with \verb|icx -fiopenmp -fopenmp-targets=<target triple>|. Intel has support for \verb|spir64| and \verb|spirv64| architectures, as well as the above.
\item NVIDIA on Power with IBM Open XL C/C++ for Linux on Power: \verb|xlc –qsmp –qoffload –qtgtarch=sm_70|
\end{itemize} 

We refer to:
\url{https://www.archer.ac.uk/training/course-material/2019/06/AdvOpenMP-manch/L10-OpenMPTargetOffload.pdf}
from which we took the first example and to \url{https://www.openmp.org/wp-content/uploads/2021-10-20-Webinar-OpenMP-Offload-Programming-Introduction.pdf} from which we took the second example. 

\section{Synchronisation Primitives}

\subsection{Barrier}

OpenMP provides a straightforward, explicit barrier construct:

\raggedbottom
\begin{codebox}[]{\href{https://godbolt.org/z/qK53jYonK}{\ExternalLink}}
\footnotesize An example of the use of a barrier.
\tcblower
\cppfile{code_examples/openmp/barrier1.cpp}
\end{codebox}

Especially with nested parallel regions, the behaviour can be quite non-trivial.
All threads of the current team must complete execution of all tasks bound to the same parallel region prior to continuing past the barrier. What is the current team, however, e.g., whether it is created by the innermost enclosing parallel region,
depends on the settings of the nesting:

\raggedbottom
\begin{codebox}[]{\href{https://godbolt.org/z/eb7n975ea}{\ExternalLink}}
\footnotesize An example of the use of a barrier, whose study is left as an exercise.
\tcblower
\cppfile{code_examples/openmp/barrier2.cpp}
\end{codebox}

Barrier is also implied by the entry and exit in \cpp{parallel} regions.
There is also an implicit barrier at the end of a \cpp{for}, \cpp{sections}, \cpp{scope}, and \cpp{workshare} constructs, unless one explicitly adds a \cpp{nowait} clause.

\subsection{Fences and Flushes}

Close to a memory fence, the \cpp{flush} construct provides point at which a thread is guaranteed to
have a consistent view of memory, similar to a fence. 

The \cpp{flush} is also implied by the entry and exit in parallel regions, critical regions, operations with locks etc. 

\subsection{Atomic Variables}

OpenMP has a rich support for atomic variables. 
At the simplest one may consider:

\raggedbottom
\begin{codebox}[]{\href{https://godbolt.org/z/baqqEMKEe}{\ExternalLink}}
\footnotesize An example of the use of atomic variables.
\tcblower
\cppfile{code_examples/openmp/atomic1.cpp}
\end{codebox}

More broadly, one can specify:
\begin{itemize}
\item operations for which the atomicity is enforced, out of \cpp{read},  \cpp{write}, \cpp{update}, \cpp{capture}, out of which update is the default. Capture makes it possible to use operators such as \cpp{+=}, e.g., \cpp{{v = x; x binop= expr;} }.
\item memory order, out of \cpp{seq_cst}, \cpp{acq_rel}, \cpp{release},  \cpp{acquire},  \cpp{relaxed} as discussed in Chapter \ref{ch:concepts}. The default is relaxed-consistency shared memory model.
\end{itemize}

E.g. \cpp{#pragma omp atomic write} or \cpp{#pragma omp atomic seq_cst}.
Note that you need to use \cpp{{}} to create a new block, wherein the operations would be atomic, if you wish to consider some sequence of read and write operations. 

\subsection{Reductions}

As we have briefly mentioned on page \pageref{sec:reduction1}, one may consider a reduction, which produces a single value from an associative operations such as addition, multiplication, taking of the minimum, maximum, or custom associative functions. The goal is for each thread to run the reduction with a private copy and then to produce the final result with the same reduction, perhaps in a hierarchical fashion. A simplistic blueprint is provided in the following example:

\raggedbottom
\begin{codebox}[]{\href{https://godbolt.org/z/orfM3x13M}{\ExternalLink}}
\footnotesize An example of the use of reductions.
\tcblower
\cppfile{code_examples/openmp/reduction.cpp}
\end{codebox}

\subsection{Mutexes}

OpenMP has only a limited support for mutexes, as in does not support any RAII variant, 
which makes them hard to use correctly. One can, however, construct one own's RAII variant:

%\raggedbottom
%\begin{codebox}[]{\href{https://godbolt.org/z/}{\ExternalLink}}
%\footnotesize An example of the use of OpenMP mutexes.
%\tcblower
%\cppfile{code_examples/openmp/mutex1.cpp}
%\end{codebox}

\raggedbottom
\begin{codebox}[]{\href{https://godbolt.org/z/vnsW7v35z}{\ExternalLink}}
\footnotesize An example of the use of OpenMP mutexes.
\tcblower
\cppfile{code_examples/openmp/mutex2.cpp}
\end{codebox}

\subsection{Critical Sections}

Instead of mutexes, one can safely use critical sections in OpenMP. 
A critical section is a block of code which can be executed by at most 
one thread at one time.
This can be used to protect non-trivial non-associative update operations, 
for which we cannot use reductions: 

\raggedbottom
\begin{codebox}[]{\href{https://godbolt.org/z/YYeqYEdv3}{\ExternalLink}}
\footnotesize An example of the use of a critical section.
\tcblower
\cppfile{code_examples/openmp/critical1.cpp}
\end{codebox}

The use of critical sections does come with a substantial penalty in terms of performance, though. 

\section{Algorithms}

While OpenMP does not really implement any algorithms, some of the data-parallel constructs are similar to algorithms in the STL library. Notably:

\subsection{For Each}

OpenMP makes it possible to use ``for each'':

\raggedbottom
\begin{codebox}[]{\href{https://godbolt.org/z/}{\ExternalLink}}
\footnotesize An example of the use of for each.
\tcblower
\cppfile{code_examples/openmp/for_each1.cpp}
\end{codebox}

\raggedbottom
\begin{codebox}[]{\href{https://godbolt.org/z/enK6dYEoW}{\ExternalLink}}
\footnotesize An example of the use of for each.
\tcblower
\cppfile{code_examples/openmp/for_each2.cpp}
\end{codebox}

\section{Exercises}
\label{sec:}

A useful exercise may be to consider a simple ray tracer \url{https://github.com/ssloy/tinyraytracer/blob/master/tinyraytracer.cpp} with basic OpenMP parallelization and to try to target a GPGPU. 


%\subsection{Reduce}
%\subsection{Merge}

