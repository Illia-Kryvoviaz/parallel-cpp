\chapter{Use Case 2: Sorting}

%In the first chapter, we have seen that within shared-memory parallel programming, we have broadly %four options:
%\begin{itemize}
%\item Confinement: Do not share memory between threads. 
%\item Immutability: Do not share any mutable data between threads. 
%\item Thread-safe code: Use data types with additional guarantees for storing any mutable data shared between threads, or even better, use implementations of algorithms that are already parallelized and handle the concurrency issues for you. 
%\item Synchronization: Use synchronization primitives to prevent accessing the variable at the same time. 
%\end{itemize}

In the first chapter, we have seen that within 
shared-memory parallel programming, we have broadly 
four options Confinement, Immutability, 
Use of Thread-safe Code, and Synchronization. 
Let us now explore the latter two options on the use case of sorting. As Table \ref{ch08:examples} suggests, sorting is Nick's class (cf. p. \pageref{sec:Nicks}): for $n$ items, we can consider $O(n^2)$ pairs of items in parallel, compare them to obtain a binary value, and then for each item, obtain its rank in the sorted order by adding the binary values. This is known as the parallel ranking. 

There are many other parallel algorithms based on picking minimum and maximum from a small set, as well as algorithms based on hashing.

\begin{table}[h!]
\begin{tabular}{lll}
Algorithm                          & $p(n)$ Processors  & Time \\
\hline
Sequential algorithms              & 1                  & $O(n \log n)$ \\
Parallel divide and conquer        & $O(1)$             & $O(n \log n)$ \\
                                   & $O(\log n)$        & $O\left(\frac{n(\log n)}{p(n)}\right)$ \\
                                   & $\omega(\log n)$   & $O(n)$ \\
Parallel Ranking                   & $O(n^2)            & $O(\log n)$
\end{tabular}
\caption{Parallelisability of common sorting algorithms, following \cite{greenlaw1995limits}.}
\label{ch08:examples}
\end{table}


\section{Thread-safe Code in C++20}

A prime example of the use of thread-safe code in C++ is the standard template library (STL) with a suitable execution policy. We have seen that the header \cpp{execution} defines objects \cpp{std::execution::seq}, \cpp{std::execution::par}, \cpp{std::execution::par_unseq}, which can be passed as the first argument of any standard algorithm, e.g., \\
\cpp{std::vector<int> my_data; std::sort(std::execution::par,my_data.begin(),my_data.end());}

\begin{codebox}[breakable]{}
\footnotesize Example of parallel sorting in STL.
\tcblower
\cppfile{code_examples/sort/sort_first.cpp}
\end{codebox}

Now, knowing what your implementation of standard template library (STL) would actually use is non-trivial. One may imagine a code such as:

\begin{codebox}[breakable]{}
\footnotesize Example of a quick sort using STL.
\tcblower
\cppfile{code_examples/sort/sort_qs_stl.h}
\end{codebox}

In the Intel Thread Building Blocks (TBB) backend of the GCC implementation, you would actually find: 
% \href{https://github.com/gcc-mirror/gcc/blob/16e2427f50c208dfe07d07f18009969502c25dc8/libstdc%2B%2B-v3/include/pstl/algorithm_impl.h#L2107}{\ExternalLink}

\begin{codebox}[breakable]{}
\footnotesize Example of parallel sorting implementation in STL.
\tcblower
\cppfile{code_examples/sort/sort_gcc_impl.h}
\end{codebox}

This uses a ``sorting based on task tree and parallel merge'', while making use of several non-trivial tricks, including \cpp{tbb::task_scheduler_init}, \cpp{std::thread::hardware_concurrency()}, and \cpp{std::hardware_constructive_interference_size}. (Contrast this with the serial version of GCC sort, which uses a multi-way mergesort, and GCC stable\_sort, which uses a quicksort.) We wish to make use of the STL, rather than redevelop it, in the first instance. 

% https://en.cppreference.com/w/cpp/thread/hardware_destructive_interference_size

Even making full use of the STL is quite non-trivial. Here, we present an overview of the sorting-related routines in verbatim from the fantastic book ``A Complete Guide to Standard C++ Algorithms'' of Simon Toth \cite{toth2023}, in compliance with the license. 

\subsection{\cpp{strict_weak_ordering}}

First, however, let us introduce the \cpp{strict_weak_ordering} ``comparator'' required, following the material of Simon Toth.
Implementing a \cpp{strict_weak_ordering} for a custom type at minimum requires providing an overload of \cpp{operator<} with the following behaviour:

\begin{itemize}
    \item irreflexive $\neg f(a,a)$
    \item anti-symmetric $f(a,b) \Rightarrow \neg f(b,a)$
    \item transitive $(f(a,b) \wedge f(b,c)) \Rightarrow f(a,c)$
\end{itemize}

A good default for a \cpp{strict_weak_ordering} implementation is lexicographical ordering. Lexicographical ordering is also the ordering provided by standard containers.
C++20 introduced the spaceship operator, whereby user-defined types can easily access the default version of lexicographical ordering.

\emph{C++20 does not enforce that the comparator is thread safe, although that matters greatly for the thread-safe nature of the whole sorting algorithm!}

\begin{codebox}[breakable]{\href{https://compiler-explorer.com/z/GvMGhYdos}{\ExternalLink}}
\footnotesize Example of three approaches to implementing lexicographical comparison for a custom type \emph{that are not thread safe!}.
\tcblower
\cppfile{code_examples/theory/comparable_code.h}
\end{codebox}

The default lexicographical ordering (line 14) works recursively. It starts with the object’s bases first, left-to-right, depth-first and then non-static members in declaration order (processing arrays element by element, left-to-right).

The type returned for the spaceship operator is the common comparison category type for the bases and members, one of:
\begin{itemize}
    \item \cpp{std::strong_ordering}
    \item \cpp{std::weak_ordering}
    \item \cpp{std::partial_ordering}
\end{itemize}
\index{\cpp{std::strong_ordering}}
\index{\cpp{std::weak_ordering}}
\index{\cpp{std::partial_ordering}}

\subsection{\texorpdfstring{\cpp{std::lexicographical_compare}}{\texttt{std::lexicographical\_compare}}}
\index{\cpp{std::lexicographical_compare}}

Lexicographical \texttt{strict\_weak\_ordering} for ranges is exposed through the \newline\texttt{std::lexicographical\_compare} algorithm.

\cppversions{\texttt{lex\dots compare}}{\CC98}{\CC20}{\CC17}{\CC20}

\constraints{(\cpp{input_range}, \cpp{input_range})}{(\cpp{forward_range}, \cpp{forward_range})}{\cpp{operator<}}{\cpp{strict_weak_ordering}}

\begin{codebox}[breakable]{\href{https://compiler-explorer.com/z/38vdfe8vo}{\ExternalLink}}
\footnotesize Example of using \cpp{lexicographical_compare} and the built-in less than operator to compare vectors of integers.
\tcblower
\cppfile{code_examples/algorithms/lexicographical_compare_code.h}
\end{codebox}

Because the standard containers already offer a built-in lexicographical comparison, the algorithm mainly finds use for comparing raw C arrays and in cases when we need to specify a custom comparator.

\begin{codebox}[]{\href{https://compiler-explorer.com/z/Ydedor1cr}{\ExternalLink}}
\footnotesize Example of using \cpp{lexicographical_compare} for C-style arrays and customizing the comparator.
\tcblower
\cppfile{code_examples/algorithms/lexicographical_compare_useful_code.h}
\end{codebox}

\subsection{\texorpdfstring{\cpp{std::lexicographical_compare_three_way}}{\texttt{std::lexicographical\_compare\_three\_way}}}
\index{\cpp{std::lexicographical_compare_three_way}}

The \texttt{std::lexicographical\_compare\_three\_way} is the spaceship operator equivalent to \texttt{std::lexicographical\_compare}. It returns one of:
\begin{itemize}
    \item\texttt{std::strong\_ordering}
    \item \texttt{std::weak\_ordering}
    \item \texttt{std::partial\_ordering}
\end{itemize}

The type depends on the type returned by the elements' spaceship operator.

\cppversions{\texttt{lex\dots three\_way}}{\CC20}{\CC20}{N/A}{N/A}
\constraints{\texttt{(input\_range, input\_range)}}{}{\texttt{operator<=>}}{\texttt{strong\_ordering}, \texttt{weak\_ordering}, \texttt{partial\_ordering}}

\begin{codebox}[]{\href{https://compiler-explorer.com/z/a73njbW4M}{\ExternalLink}}
\footnotesize Example of using \cpp{std::lexicographical_compare_three_way}.
\tcblower
\cppfile{code_examples/algorithms/lexicographical_compare_three_way_code.h}
\end{codebox}

\subsection{\texorpdfstring{\cpp{std::sort}}{\texttt{std::sort}}}
\index{\cpp{std::sort}}

The \cpp{std::sort} algorithm is the canonical $O(n\log n)$ sort.

\cppversions{\texttt{sort}}{\CC98}{\CC20}{\CC17}{\CC20}
\constraints{\cpp{random_access_range}}{\cpp{random_access_range}}{\cpp{operator<}}{\cpp{strict_weak_ordering}}

Due to the $O(n\log n)$ complexity guarantee, \cpp{std::sort} only operates on \cpp{random_access} ranges. Notably, \cpp{std::list} offers a method with approximately $n\log n$ complexity.

\begin{codebox}[]{\href{https://compiler-explorer.com/z/W9WodGs71}{\ExternalLink}}
\footnotesize Basic example of using \cpp{std::sort} and \cpp{std::list::sort}.
\tcblower
\cppfile{code_examples/algorithms/sort_code.h}
\end{codebox}

With C++20, we can take advantage of projections to sort by a method or member:

\begin{codebox}[breakable]{\href{https://compiler-explorer.com/z/35YK5q15M}{\ExternalLink}}
\footnotesize Example of using a projection in conjunction with a range algorithm. The algorithm will sort the elements based on the values obtained by invoking the method \cpp{value} on each element.
\tcblower
\cppfile{code_examples/algorithms/sort_projection_code.h}
\end{codebox}

Before C++14, you would have to fully specify the type of the comparator, i.e. \cpp{std::greater<double>{}}. The type erased variant \cpp{std::greater<>{}} relies on type deduction to determine the parameter types. Projections accept an unary invocable, including pointers to members and member functions.

\subsection{\texorpdfstring{\cpp{std::stable_sort}}{\texttt{std::stable\_sort}}}
\index{\cpp{std::stable_sort}}

The \cpp{std::sort} is free to re-arrange equivalent elements, which can be undesirable when re-sorting an already sorted range. The \cpp{std::stable_sort} provides the additional guarantee of preserving the relative order of equal elements.

\cppversions{\texttt{stable\_sort}}{\CC98}{N/A}{\CC17}{\CC20}
\constraints{\cpp{random_access_range}}{}{\cpp{operator<}}{\cpp{strict_weak_ordering}}

If additional memory is available, stable\_sort remains $O(n\log n)$. However, if it fails to allocate, it will degrade to an $O(n\log n\log n)$ algorithm.

\begin{codebox}[]{\href{https://compiler-explorer.com/z/bh7o1KG91}{\ExternalLink}}
\footnotesize Example of re-sorting a range using \cpp{std::stable_sort}, resulting in a guaranteed order of elements.
\tcblower
\cppfile{code_examples/algorithms/stable_sort_code.h}
\end{codebox}

\subsection{\texorpdfstring{\cpp{std::is_sorted}}{\texttt{std::is\_sorted}}}
\index{\cpp{std::is_sorted}}

The \cpp{std::is_sorted} algorithm is a linear check returning a boolean denoting whether the ranges elements are in non-descending order.

\cppversions{\texttt{is\_sorted}}{\CC11}{\CC20}{\CC17}{\CC20}
\constraints{\texttt{forward\_range}}{\texttt{forward\_range}}{\cpp{std::less}}{\texttt{strict\_weak\_ordering}}

\begin{codebox}[]{\href{https://compiler-explorer.com/z/zqK9aohWs}{\ExternalLink}}
\footnotesize Example of testing a range using \cpp{std::is_sorted}.
\tcblower
\cppfile{code_examples/algorithms/is_sorted_code.h}
\end{codebox}

\subsection{\texorpdfstring{\cpp{std::is_sorted_until}}{\texttt{std::is\_sorted\_until}}}
\index{\cpp{std::is_sorted_until}}

The \cpp{std::is_sorted_until} algorithm returns the first out-of-order element in the given range, thus denoting a sorted sub-range.

\cppversions{\texttt{is\_sorted\_until}}{\CC11}{\CC20}{\CC17}{\CC20}
\constraints{\texttt{forward\_range}}{\texttt{forward\_range}}{\texttt{std::less}}{\texttt{strict\_weak\_ordering}}

\begin{codebox}[]{\href{https://compiler-explorer.com/z/48PaYE9Ej}{\ExternalLink}}
\footnotesize Example of testing a range using \cpp{std::is_sorted_until}.
\tcblower
\cppfile{code_examples/algorithms/is_sorted_until_code.h}
\end{codebox}

Note that because of the behaviour of \cpp{std::is_sorted_until}, the following is always true:\\
\begin{small}\cpp{std::is_sorted(r.begin(), std::is_sorted_until(r.begin(), r.end()))}\end{small}

\subsection{\texorpdfstring{\cpp{std::partial_sort}}{\texttt{std::partial\_sort}}}
\index{\cpp{std::partial_sort}}

The \cpp{std::partial_sort} algorithm reorders the range's elements such that the leading sub-range is in the same order it would when fully sorted. However, the algorithm leaves the rest of the range in an unspecified order.

\cppversions{\texttt{partial\_sort}}{\CC98}{\CC20}{\CC17}{\CC20}

\constraints{\cpp{(random_access_range, random_access_iterator)}}{\cpp{(random_access_range, random_access_iterator)}}{\cpp{operator<}}{\texttt{strict\_weak\_ordering}}

The benefit of using a partial sort is faster runtime — approximately $O(n \log k)$, where k is the number of elements sorted.

\begin{codebox}[]{\href{https://compiler-explorer.com/z/Th9YTzGsM}{\ExternalLink}}
\footnotesize Example of using \cpp{std::partial_sort} to sort the first three elements of a range.
\tcblower
\cppfile{code_examples/algorithms/partial_sort_code.h}
\end{codebox}

\subsection{\texorpdfstring{\cpp{std::partial_sort_copy}}{\texttt{std::partial\_sort\_copy}}}
\index{\cpp{std::partial_sort_copy}}

The \cpp{std::partial_sort_copy} algorithm has the same behaviour as \linebreak\cpp{std::partial_sort}; however, it does not operate inline. Instead, the algorithm writes the results to a second range.

\cppversions{\texttt{partial\_sort\_copy}}{\CC98}{\CC20}{\CC17}{\CC20}
\constraints{\cpp{input_range -> random_access_range}}{\cpp{forward_range -> random_access_range}}{\cpp{operator<}}{\texttt{strict\_weak\_ordering}}

The consequence of writing output to a second range is that the source range does not have to be mutable nor provide random access.

\begin{codebox}[]{\href{https://compiler-explorer.com/z/Ehjbhfj9s}{\ExternalLink}}
\footnotesize Example of using \cpp{std::partial_sort_copy} to iterate over ten integers read from standard input and storing the top three values in sorted order.
\tcblower
\cppfile{code_examples/algorithms/partial_sort_copy_code.h}
\end{codebox}

\section{Synchronization}

Let us now use synchronization primitives to implement simple sorting algorithms. 

\subsection{Quick sort with user-level threads}

For instance, let us consider quick-sort, for a start:

\begin{codebox}[breakable]{}
\footnotesize An example of parallel quick sort in OpenMP.
\tcblower
\cppfile{code_examples/sort/sort_qs.h}
\end{codebox}

Intel suggests a three-way quick-sort (\url{https://software.intel.com/content/www/us/en/develop/articles/an-efficient-parallel-three-way-quicksort-using-intel-c-compiler-and-openmp-45-library.html
}) using the task construct:

\begin{codebox}[breakable]{}
\footnotesize An example of three-way parallel quick sort in OpenMP.
\tcblower
\cppfile{code_examples/sort/sort_qs3w.h}
\end{codebox}

\subsection{Merge sort with user-level threads}

Similarly, one could parallelize merge sort. Let us consider a simple merge sort:

\begin{codebox}[breakable]{}
    \footnotesize An example of merge sort.
    \tcblower
    \cppfile{code_examples/sort/sort_merge1.h}
    \end{codebox}

Using the task construct in OpenMP, we can parallelize this as:

    \begin{codebox}[breakable]{}
        \footnotesize An example of merge sort in OpenMP.
        \tcblower
        \cppfile{code_examples/sort/sort_merge2.h}
        \end{codebox}

These examples of quick sort and merge sort are actually competitive. 
 On GPGPUs, one may wish to consider Odd-Even Merge Sort:

 \begin{codebox}[breakable]{}
    \footnotesize An example of odd-even merge sort.
    \tcblower
    \cppfile{code_examples/sort/sort_merge3.h}
    \end{codebox}

which could be vectorized. Notice that many current compilers are able to vectorize 
the code for you, without the need to use intrinsics: 

\begin{codebox}[breakable]{}
    \footnotesize An example of vectorized subroutine of the odd-even merge sort.
    \tcblower
    \cppfile{code_examples/sort/sort_merge4.h}
    \end{codebox}

    \begin{codebox}[breakable]{}
        \footnotesize An example of vectorized subroutine of the odd-even merge sort.
        \tcblower
        \cppfile{code_examples/sort/sort_merge5.h}
        \end{codebox}

See \url{https://xhad1234.github.io/Parallel-Sort-Merge-Join-in-Peloton/} for a state-of-the-art implementation. 

\subsection{Bubble sort with vectorisation}

One may wish to ``do better'' by considering alternative algorithms and alternative synchronisation primitives. Quite possibly the simplest sorting algorithm is the bubble sort: 

\begin{codebox}[breakable]{}
\footnotesize An example of bubble sort.
\tcblower
\cppfile{code_examples/sort/sort_bubble1.h}
\end{codebox}

This can be parallelized with OpenMP:

\begin{codebox}[breakable]{}
\footnotesize An example of bubble sort in OpenMP.
\tcblower
\cppfile{code_examples/sort/sort_bubble2.h}
\end{codebox}

\subsection{Sample sort}
 
% GCC C++ PSTL, Intel Thread Building Blocks: “sorting based on task tree and parallel merge”

Another alternative is to consider sample sort. This is implemented in Boost 
(\url{https://www.boost.org/doc/libs/develop/libs/sort/doc/html/sort/parallel.html}), 
and as with many ideas implemented in Boost, this is a great idea. 


%Vektorizované algoritmy založené na výběru minima a maxima
%Algoritmy založené na hashování
